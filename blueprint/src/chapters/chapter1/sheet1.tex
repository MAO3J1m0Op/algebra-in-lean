\section{Defining the Group}

\begin{definition}[Group]
    \label{definition : Group}
    \lean{Defs.Group}
    \leanok
    \uses{definition : Monoid}
    A group is a set $G$ with the following properties:
    \begin{itemize}
        \item It must have a binary operation $\mu$.
        \item This operation must be associative, so $\forall a, b, c \in G, \mu(\mu(a, b), c) = \mu(a, \mu(b, c))$
        \item $G$ must contain an identity $\mathbb{e}$ where $\forall a \in G, \mu (a, \mathbb{e}) = \mu(\mathbb{e}, a) = a$
        \item There must also be an inverse function $\iota$, where $\forall a \in G, \mu(\iota(a), a) = \mathbb{e}$
    \end{itemize}
\end{definition}

\begin{theorem}
    \label{theorem : inv_op}
    \lean{Defs.Sheet1.op_inv}
    \leanok
    \uses{definition : Group}
    Applying the operation to an element and it's inverse returns the identity: $\forall a \in G, \mu(a, \iota(a)) = \mathbb{e}$
\end{theorem}
