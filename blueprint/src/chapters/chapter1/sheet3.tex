\section{Uniqueness Theorems}

\begin{theorem}
    \label{theorem : id_unique}
    \lean{AlgebraInLean.id_unique}
    \leanok
    \uses{definition : Monoid}
    In a monoid, the identity element is unique:
    $\forall a \in G, (\forall b \in G, \mu(a, b) = \mu(b, a) = b) \Rightarrow a = \mathbb{e}$
  \end{theorem}

\begin{theorem}
    \label{theorem : inv_unique}
    \lean{AlgebraInLean.inv_unique}
    \leanok
    \uses{definition : Group}
    In a group, each element has a unique inverse:
    $\forall a, b \in G, (\mu(a, b) = \mu(b, a) = \mathbb{e}) \Rightarrow b = \iota(a)$
  \end{theorem}

\begin{theorem}
  \label{theorem : inv_anticomm}
  \lean{AlgebraInLean.inv_anticomm}
  \leanok
  \uses{theorem : inv_unique, theorem : op_inv}
  The inverse of a product is the product of the inverses in reverse order:
  $\forall a, b \in G, \iota(\mu(a, b)) = \mu(\iota(b), \iota(a))$
\end{theorem}

\begin{theorem}
  \label{theorem : inv_inv}
  \lean{AlgebraInLean.inv_inv}
  \leanok
  \uses{theorem : inv_unique, theorem : op_inv}
  The inverse of the inverse of an element is that element:
  $\forall a \in G, \iota(\iota(a)) = a$
\end{theorem}

\begin{theorem}
  \label{definition : right_cancel}
  \lean{AlgebraInLean.right_cancel}
  \leanok
  \uses{definition : Group}
  Right-multiplication by the same element on both sides of an equation can be cancelled.
  $\forall a, b, c \in G, \mu(b, a) = \mu(c, a) \Rightarrow b = c$
\end{theorem}
