\begin{definition}[Injective]
    \label{definition : injective}
    \lean{Defs.Interlude.Injective}
    \leanok
    Given a function \( f \) mapping from a set \( X \) to a set \( Y \):
    \[
        \forall x,~ x' \in X,~ f(x) = f(x') \implies x = x'.
    \]
    Otherwise known as a ``one-to-one'' function.
\end{definition}

\begin{definition}[Surjective]
    \label{definition : surjective}
    \lean{Defs.Interlude.Surjective}
    \leanok
    Given a function \( f \) mapping from a set \( X \) to a set \( Y \):
    \[
        \forall y \in Y, \exists x \in X, f(x) = y
    \]
    Otherwise known as an ``onto'' function.
\end{definition}

\begin{definition}[Bijective]
    \label{definition : bijective}
    \lean{Defs.Interlude.Bijective}
    \leanok
    A function that is both injective and surjective.
\end{definition}

\begin{theorem}[Composition of surjective functions]
    \label{theorem : surjective_comp}
    \lean{Defs.Interlude.surjective_comp}
    \leanok
    WIP
\end{theorem}

\begin{proof}
    \leanok
    \uses{}
    This obviously follows from what we did so far.
\end{proof}

\begin{theorem}[Composition of bijective functions]
    \label{theorem : bijective_comp}
    \lean{Defs.Interlude.bijective_comp}
    \leanok
    WIP
\end{theorem}

\begin{proof}
    \leanok
    \uses{}
    This obviously follows from what we did so far.
\end{proof}

