\section{Intro to Morphisms}

\begin{definition}[Homomorphism]
    \label{definition : homomorphism}
    \lean{Homomorphism}
    \leanok
    Of a map \( \phi \) mapping from a group \( G \) to a group \( H \), for
    elements \( g \in G \) and \( h \in H \), \( \phi \) is a homomorphism if and only if:
    \[
        \phi(g)\phi(h) = \phi(gh)
    \]
\end{definition}

\begin{theorem}[Homomorphism_def]
    \label{theorem : homomorphism}
    \lean{Homomorphism}
    \leanok
    Of a map \( \phi \) mapping from a group \( G \) to a group \( H \), for
    elements \( g \in G \) and \( h \in H \), \( \phi \) is a homomorphism if and only if:
    \[
        \phi(g)\phi(h) = \phi(gh)
    \]
\end{definition}

\begin{proof}
    \leanok
    \uses{}
    This obviously follows from what we did so far.
\end{proof}

\begin{definition}[Isomorphism]
    \label{definition : isomorphism}
    \lean{Isomorphism}
    \leanok
\end{definition}

\begin{theorem}[Suppose \( \phi : G \rightarrow H \) is a homomorphism. Then,
\( \phi(e) = e \).]
    \label{theorem : hom_id_to_id}
    \lean{hom_id_to_id}
    \leanok
    Of a map \( \phi \) mapping from a group \( G \) to a group \( H \), for
    elements \( g \in G \) and \( h \in H \), \( \phi \) is a homomorphism if and only if:
    \[
        \phi(g)\phi(h) = \phi(gh)
    \]
\end{definition}

\begin{proof}
    \leanok
    \uses{}
    This obviously follows from what we did so far.
\end{proof}

\begin{theorem}[For all \( a, b \in G, ab = e \implies b = a^{-1}]
    \label{theorem : two_sided_inv}
    \lean{two_sided_inv}
    \leanok
\end{definition}

\begin{proof}
    \leanok
    \uses{}
    This obviously follows from what we did so far.
\end{proof}

\begin{theorem}[Suppose \( \phi : G \rightarrow H \) is a homomorphism. If \(g \in G \), then \( \phi(g^{-1}) = \phi(g)^{-1} \).]
    \label{theorem : hom_inv_to_inv}
    \lean{hom_inv_to_inv}
    \leanok
\end{definition}

\begin{proof}
    \leanok
    \uses{}
    This obviously follows from what we did so far.
\end{proof}
