\section{Intro to Morphisms}

\begin{definition}[Homomorphism]
    \label{definition : Homomorphism}
    \lean{Homomorphism}
    \leanok
    Of a map \( \phi \) mapping from a group \( G \) to a group \( H \), for
    elements \( g \in G \) and \( h \in H \), \( \phi \) is a
    \emph{homomorphism} if and only if:
    \[
        \phi(g)\phi(h) = \phi(gh)
    \]
\end{definition}

\begin{definition}[Isomorphism]
    \label{definition : Isomorphism}
    \lean{Isomorphism}
    \leanok
    Of a map \( \phi \) mapping from a group \( G \) to a group \( H \), \( \phi
    \) is an \emph{isomorphism} if and only if it is a bijective homomorphism.
\end{definition}

\begin{theorem}
    \label{theorem : hom_id_to_id}
    \lean{hom_id_to_id}
    \leanok
    Suppose \( \phi : G \rightarrow H \) is a homomorphism. Then \( \phi(e) = e \).
\end{theorem}

\begin{theorem}
    \label{theorem : hom_inv_to_inv}
    \lean{hom_inv_to_inv}
    \leanok
    Suppose \( \phi : G \rightarrow H \) is a homomorphism. If \(g \in G \), then \( \phi(g^{-1}) = \phi(g)^{-1} \).
\end{theorem}
