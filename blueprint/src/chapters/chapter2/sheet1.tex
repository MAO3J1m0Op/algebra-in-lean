\section{Intro to Morphisms}

\begin{definition}[Homomorphism]
    \label{definition : Homomorphism}
    \lean{AlgebraInLean.Homomorphism}
    \leanok
    \uses{definition : Group}
    A homomorphism is a map \( \phi \) from a group \( G \) to a group \( H \), for
    elements \( g,~ h \in G \), \( \phi \) is a
    \emph{homomorphism} if and only if:
    \[
        \mu(\phi(g),\phi(h)) = \phi(\mu(g,h))
    \]
\end{definition}

\begin{definition}[Isomorphism]
    \label{definition : Isomorphism}
    \lean{AlgebraInLean.Isomorphism}
    \leanok
    \uses{definition : Group, definition : Homomorphism, definition : Bijective}
    An \emph{isomorphism} \( \phi \) from a group \( G \) to a group \( H \) is a bijective homomorphism.
\end{definition}

\begin{theorem}
    \label{theorem : hom_id_to_id}
    \lean{AlgebraInLean.hom_id_to_id}
    \leanok
    \uses{definition : Group, definition : Homomorphism}
    Suppose \( \phi : G \rightarrow H \) is a homomorphism. Then \(
    \phi(\mathbb e) = \mathbb e \).
\end{theorem}

\begin{theorem}
    \label{theorem : hom_inv_to_inv}
    \lean{AlgebraInLean.hom_inv_to_inv}
    \leanok
    \uses{definition : Group, definition : Homomorphism}
    Suppose \( \phi : G \rightarrow H \) is a homomorphism. If \(g \in G \),
    then \( \phi(\iota(g)) = \iota(\phi(g)) \).
\end{theorem}
