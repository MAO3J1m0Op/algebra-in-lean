\section{Automorphisms}

\begin{definition}[Endomorphism]
    \label{definition : Endomorphism}
    \lean{AlgebraInLean.Endomorphism}
    \leanok
    \uses{definition : Group, definition : Homomorphism}
    An \emph{endomorphism} is a homomorphism \( \phi \) mapping from a group \( G \) to itself.
\end{definition}

\begin{definition}[Automorphism]
    \label{definition : Automorphism}
    \lean{AlgebraInLean.Automorphism}
    \leanok
    \uses{definition : Group, definition : Isomorphism, definition :
    Endomorphism}
    An \emph{automorphism} is a bijective endomorphism; i.e.~an isomorphism
    from a group \( G \) to itself.
\end{definition}

\begin{definition}[conjugate]
    \label{definition : conjugate}
    \lean{AlgebraInLean.conjugate}
    \leanok
    \uses{definition : Group}
    Two elements \( a,~b \) of a group \( G \) are \emph{conjugates} if there
    exists another element \( g \) in \( G \) such that \( b = gag^{-1} \) (i.e.,~ \(
    \mu(\mu(g,~a),~\iota(g)) \) ).
\end{definition}

\begin{theorem}
    \label{theorem : conj_automorphism}
    \lean{AlgebraInLean.conj_automorphism}
    \leanok
    \uses{definition : Group, definition : Automorphism}
    Given a group \( G \) and an element \( a \in G \), conjugation by \( a \)
    is an automorphism.
\end{theorem}
