\section{Generators and the Cyclic Group}

\begin{theorem}
    \label{theorem : subgroupOrder}
    \lean{AlgebraInLean.Subgroup.subgroupOrder}
    \leanok
    \uses{definition : Subgroup}
    Inclusion ($\subset$) forms a partial order over the set of all subgroups of $G$.
\end{theorem}

\begin{theorem}
    \label{theorem : Minimal_smallest}
    \lean{AlgebraInLean.Subgroups.Minimal_smallest}
    \leanok
    \uses{definition : Subgroup_Minimal, theorem : subgroupOrder}
    Every subgroup $H$ has order greater than or equal to the minimal subgroup.
\end{theorem}

\begin{theorem}
    \label{theorem : Maximal_largest}
    \lean{AlgebraInLean.Subgroups.Maximal_largest}
    \leanok
    \uses{definition : Subgroup_Maximal, theorem : subgroupOrder}
    Every subgroup $H$ has order less than or equal to the maximal subgroup.
\end{theorem}

\begin{theorem}
    \label{definition : subgroup_eq_Maximal_of_card_eq_G}
    \lean{AlgebraInLean.Subgroups.subgroup_eq_Maximal_of_card_eq_G}
    \uses{theorem : subgroupOrder, }
    If $G$ is a finite group, and $H$ is a subgroup of $G$ with $|H| = |G|$, then $H = G$.
\end{theorem}

\begin{definition}
    \label{definition : Generate}
    \lean{AlgebraInLean.Subgroups.Generate}
    \leanok
    \uses{definition : Subgroup}
    The subgroup of $G$ generated by a subset $S \subset G$ is defined to be the smallest subgroup containing all of $S$. Notated $\langle S \rangle$, it is written as

    \begin{equation*}
        \langle S \rangle := \bigcap_{\genfrac{}{}{0pt}{}{G \le H}{S \subset H}} H
    \end{equation*}
\end{definition}

\begin{theorem}
    \label{theorem : Generate_empty}
    \lean{AlgebraInLean.Subgroup.Generate_empty}
    \leanok
    \uses{definition : Generate}
    $\langle \emptyset \rangle = \{e\}$, the minimal subgroup.
\end{theorem}

\begin{theorem}
    \label{theorem : Generate_contain_set}
    \lean{AlgebraInLean.Subgroup.Generate_contain_set}
    \leanok
    \uses{definition : Generate}
    $S \subset \langle S \rangle$.
\end{theorem}

\begin{theorem}
    \label{theorem : Generate_self_eq_self}
    \lean{AlgebraInLean.Subgroup.Generate_self_eq_self}
    \leanok
    \uses{definition : Generate}
    If $H$ is a subgroup of $G$, then $\langle H \rangle = H$.
\end{theorem}

\begin{definition}
    \label{definition : Pows}
    \lean{AlgebraInLean.Subgroups.Pows}
    \leanok
    \uses{definition : Subgroup, definition : gpow}
    For a group $G$, the \textit{power subgroup} of $x \in G$ is defined as

    $$\text{Pows}(x) := \left\{ x^n \middle| n \in \mathbb{Z} \right\}$$
\end{definition}

For the following theorems, let $G$ be a group and let $x \in G$.

\begin{theorem}
    \label{theorem : Pows_eq_Generate_singleton}
    \lean{AlgebraInLean.Subgroups.Pows_eq_Generate_singleton}
    \leanok
    \uses{definition : Pows, definition : Generate}
    $\text{Pows}(x) = \langle \{x\} \rangle$.
\end{theorem}

\begin{theorem}
    \label{theorem : Pows_card_eq_order}
    \lean{AlgebraInLean.Subgroups.Pows_card_eq_order}
    \leanok
    \uses{definition : Pows, definition : order}
    $|\text{Pows}(x)| = |x|$.
\end{theorem}
