\section{Generators and the Cyclic Group}

\begin{theorem}
    \label{theorem : PartialOrder (Subgroup G)}
    % TODO : name partial order so it can be included in the blueprint
    Inclusion ($\subset$) forms a partial order over the set of all subgroups of $G$.
\end{theorem}

\begin{theorem}
    \label{theorem : Minimal_smallest}
    \lean{Defs.Subgroups.Minimal_smallest}
    \leanok
    Every subgroup $H$ has order greater than or equal to the minimal subgroup.
\end{theorem}

\begin{theorem}
    \label{theorem : Maximal_largest}
    \lean{Defs.Subgroups.Maximal_largest}
    \leanok
    Every subgroup $H$ has order less than or equal to the maximal subgroup.
\end{theorem}

\begin{theorem}
    \label{definition : Intersect}
    \lean{Defs.Subgroups.Intersect}
    \leanok
    The intersection of subgroups is itself a subgroup.
\end{theorem}

\begin{theorem}
    \label{definition : subgroup_eq_Maximal_of_card_eq_G}
    \lean{Defs.Subgroups.subgroup_eq_Maximal_of_card_eq_G}
    If $G$ is a finite group, and $H$ is a subgroup of $G$ with $|H| = |G|$, then $H = G$.
\end{theorem}

\begin{theorem}
    \label{theorem : inter_comm}
    \lean{Defs.Subgroups.inter_comm}
    The intersection of subgroups is commutative, that is $H \cap K = K \cap H$ for subgroups $H$ and $K$.
\end{theorem}

\begin{theorem}
    \label{theorem : le_intersect_self}
    \lean{Defs.Subgroups.le_intersect_self}
    For subgroups $H$ and $K$ of group $G$, $H \cap K \subset H$.
\end{theorem}

\begin{definition}
    \label{definition : Generate}
    \lean{Defs.Subgroups.Generate}
    \leanok
    \uses{definition : Subgroup}
    The subgroup of $G$ generated by a subset $S \subset G$ is defined to be the smallest subgroup containing all of $S$. Notated $\langle S \rangle$, it is written as

    \begin{equation*}
        \langle S \rangle := \bigcap_{G \le H;\; S \subset H} H
    \end{equation*}
\end{definition}

\begin{theorem}
    \label{theorem : Generate_empty}
    $\langle \emptyset \rangle = \{e\}$, the minimal subgroup.
\end{theorem}

\begin{theorem}
    \label{theorem : Generate_contain_set}
    $S \subset \langle S \rangle$.
\end{theorem}

\begin{theorem}
    \label{theorem : Generate_self_eq_self}
    If $H$ is a subgroup of $G$, then $\langle H \rangle = H$.
\end{theorem}

\begin{theorem}
    \label{theorem : Generate_smallest_closure}
    \lean{Defs.Subgroups.Generate_smallest_closure}
    \leanok
    \uses{definition : Subgroup, definition : Generate}
    $\langle S \rangle$ is the smallest subgroup containing $S$. In other terms, if $S \subset H$ and $H \le \langle S \rangle$, then $H = \langle S \rangle$.
\end{theorem}

\begin{definition}
    \label{definition : Pows}
    \lean{Defs.Subgroups.Pows}
    \leanok
    \uses{definition : Subgroup, definition : gpow}
    For a group $G$, the \textit{power subgroup} of $x \in G$ is defined as

    $$\text{Pows}(x) := \left\{ x^n \middle| n \in \mathbb{Z} \right\}$$
\end{definition}

For the following theorems, let $G$ be a group and let $x \in G$.

\begin{theorem}
    \label{theorem : Pows_contain_self}
    \lean{Defs.Subgroups.Pows_contain_self}
    \leanok
    \uses{definition : Pows}
    $x \in \text{Pows}(x)$.
\end{theorem}

\begin{theorem}
    \label{theorem : Pows_eq_Generate_singleton}
    \lean{Defs.Subgroups.Pows_eq_Generate_singleton}
    \leanok
    \uses{definition : Pows, definition : Generate}
    $\text{Pows}(x) = \langle \{x\} \rangle$.
\end{theorem}

\begin{theorem}
    \label{theorem : Pows_card_eq_order}
    \lean{Defs.Subgroups.Pows_card_eq_order}
    \leanok
    \uses{definition : Pows, definition : order}
    $|\text{Pows}(x)| = |x|$.
\end{theorem}
