\section{The Power Function}

Let $M$ be a monoid and $G$ be a group.

\begin{definition}
    \label{definition : mpow}
    \lean{Defs.Subgroups.mpow}
    \leanok
    \uses{definition : Monoid}
    For $x \in M$, the \textit{$n$th power} of $x$ (where $n \in \mathbb{N}$) is defined as the left-multiplication of $x$ with the identity of $M$ $n$ times. It is defined inductively below

    \begin{equation*}
        x^n := \begin{cases}
            e, & n = 0 \\
            \mu(x^{n-1}, x), & n > 0
        \end{cases}
    \end{equation*}
\end{definition}

\begin{definition}
    \label{definition : gpow}
    \lean{Defs.Subgroups.gpow}
    \leanok
    \uses{definition : mpow, definition : Group}
    For a group $G$ and $x \in G$, we extend the domain of the power function to include all of $\mathbb{Z}$ such that if $n < 0$, then $x^n$ is equal to $\iota(x^{-n})$. It is defined inductively below:

    \begin{equation*}
        x^n := \begin{cases}
            e, & n = 0 \\
            \mu(x^{n-1}, x), & n > 0 \\
            \iota(\mu(x^{n+1}, x)), & n < 0
        \end{cases}
    \end{equation*}
\end{definition}
