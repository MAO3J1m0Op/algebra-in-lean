\section{The Power Function}

Let $M$ be a monoid and $G$ be a group.

\begin{definition}
    \label{definition : mpow}
    \lean{Defs.Subgroups.mpow}
    \leanok
    \uses{definition : monoid}
    For $x \in M$, the \textit{$n$th power} of $x$ (where $n \in \mathbb{N}$) is defined as the left-multiplication of $x$ with the identity of $M$ $n$ times. It is defined inductively below

    \begin{equation*}
        x^n := \begin{cases}
            e, & n = 0 \\
            \mu(x^{n-1}, x), & n > 0
        \end{cases}
    \end{equation*}
\end{definition}

For the following theorems, let $x \in G$, and let $m,n \in \mathbb{N}$.

\begin{theorem}
    \label{theorem : mpow_zero}
    \lean{Defs.Subgroups.mpow_zero}
    \leanok
    \uses{definition : mpow}
    $x^0 = e$.
\end{theorem}

\begin{theorem}
    \label{theorem : mpow_succ_right}
    \lean{Defs.Subgroups.mpow_succ_right}
    \leanok
    \uses{definition : mpow}
    $x^{n+1} = \mu(x^n,x)$.
\end{theorem}

\begin{theorem}
    \label{theorem : mpow_one}
    \lean{Defs.Subgroups.mpow_one}
    \leanok
    \uses{theorem : mpow_zero}
    $x^1 = x$.
\end{theorem}

\begin{theorem}
    \label{theorem : mpow_two}
    \lean{Defs.Subgroups.mpow_two}
    \leanok
    \uses{theorem : mpow_one}
    $x^2 = \mu(x,x)$.
\end{theorem}

\begin{theorem}
    \label{theorem : mpow_succ_left}
    \lean{Defs.Subgroups.mpow_succ_left}
    \leanok
    \uses{theorem : mpow_zero, theorem : mpow_one, theorem : mpow_succ_right}
    $x^{n+1} = \mu(x,x^n)$.
\end{theorem}

\begin{theorem}
    \label{theorem : mpow_add}
    \lean{Defs.Subgroups.mpow_add}
    \leanok
    \uses{theorem : mpow_one, theorem : mpow_zero, theorem : mpow_succ_right}
    $x^{m+n} = \mu(x^m,x^n)$.
\end{theorem}

\begin{theorem}
    \label{theorem : mpow_mul}
    \lean{Defs.Subgroups.mpow_mul}
    \leanok
    \uses{theorem : mpow_add}
    $x^{mn} = (x^m)^n$.
\end{theorem}

\begin{theorem}
    \label{theorem : mpow_id}
    \lean{Defs.Subgroups.mpow_id}
    \leanok
    \uses{theorem : mpow_zero, theorem : mpow_succ_right}
    If $e$ is the identity of $M$, then $e^n = e$.
\end{theorem}

\begin{definition}
    \label{definition : gpow}
    \lean{Defs.Subgroups.gpow}
    \leanok
    \uses{definition : mpow, definition : group}
    For a group $G$ and $x \in G$, we extend the domain of the power function to include all of $\mathbb{Z}$ such that if $n < 0$, then $x^n$ is equal to $\iota(x^{-n})$. It is defined inductively below:

    \begin{equation*}
        x^n := \begin{cases}
            e, & n = 0 \\
            \mu(x^{n-1}, x), & n > 0 \\
            \iota(\mu(x^{n+1}, x)), & n < 0
        \end{cases}
    \end{equation*}
\end{definition}

For the following theorems, let $x \in G$.

\begin{theorem}
    \label{theorem : gpow_zero}
    \lean{Defs.Subgroups.gpow_zero}
    \leanok
    \uses{definition : gpow}
    $x^0 = e$.
\end{theorem}

\begin{theorem}
    \label{theorem : gpow_one}
    \lean{Defs.Subgroups.gpow_one}
    \leanok
    \uses{definition : gpow, theorem : mpow_one}
    $x^1 = x$.
\end{theorem}

\begin{theorem}
    \label{theorem : gpow_two}
    \lean{Defs.Subgroups.gpow_two}
    \leanok
    \uses{definition : gpow, theorem : mpow_two}
    $x^2 = \mu(x,x)$.
\end{theorem}

\begin{theorem}
    \label{theorem : gpow_neg_one}
    \lean{Defs.Subgroups.gpow_neg_one}
    \leanok
    \uses{definition : gpow, theorem : mpow_one}
    $x^{-1} = \iota(x)$.
\end{theorem}

\begin{theorem}
    \label{theorem : gpow_neg}
    \lean{Defs.Subgroups.gpow_neg}
    \uses{definition : gpow}
    For any $n \in \mathbb{Z}$, $x^{-n} = \iota(x^n)$.
\end{theorem}

\begin{theorem}
    \label{theorem : gpow_succ}
    \lean{Defs.Subgroups.gpow_succ}
    \uses{definition : gpow}
    For any $n \in \mathbb{Z}$, $x^{n+1} = \mu(x^n, x)$.
\end{theorem}

\begin{theorem}
    \label{theorem : gpow_pred}
    \lean{Defs.Subgroups.gpow_pred}
    \uses{definition : gpow}
    For any $n \in \mathbb{Z}$, $x^{n-1} = \mu(x^n, \iota(x))$.
\end{theorem}

\begin{theorem}
    \label{theorem : gpow_add}
    \lean{Defs.Subgroups.gpow_add}
    \uses{definition : gpow}
    For any $m,n \in \mathbb{Z}$, $x^{m+n} = \mu(x^m, x^n)$.
\end{theorem}

\begin{theorem}
    \label{theorem : gpow_sub}
    \lean{Defs.Subgroups.gpow_add}
    \leanok
    \uses{theorem : gpow_add, theorem : gpow_neg}
    For any $m,n \in \mathbb{Z}$, $x^{m-n} = \mu(x^m, \iota(x^n))$.
\end{theorem}

\begin{theorem}
    \label{theorem : gpow_mul}
    \lean{Defs.Subgroups.gpow_mul}
    \uses{definition : gpow}
    For any $m,n \in \mathbb{Z}$, $x^{mn} = (x^m)^n$.
\end{theorem}

\begin{theorem}
    \label{theorem : gpow_closure}
    \lean{Defs.Subgroups.gpow_closure}
    \leanok
    \uses{theorem : gpow_succ, theorem : gpow_pred, theorem : gpow_neg}
    Let $H$ be a subgroup of $G$. Then if $x \in H$, $x^n \in H$ for any $n \in \mathbb{Z}$.
\end{theorem}
