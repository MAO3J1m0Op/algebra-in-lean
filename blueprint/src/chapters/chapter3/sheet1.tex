\section{Defining the Subgroup}

\begin{definition}
    \label{definition : Subgroup}
    \lean{AlgebraInLean.Subgroup}
    \leanok
    \uses{definition : Group}
    Let $G$ be a group. A \textit{subgroup} $H$ of $G$ is a subset that is itself a group under the group operation of $G$. This requires that the subgroup satisfies three operations.

    \begin{enumerate}
        \item The identity of $G$ is contained within $H$.
        \item $H$ must be closed under $G$'s group operation: for any $a,b \in H$, $\mu(a,b) \in H$.
        \item $H$ must be closed under $G$'s inversion operation: for any $a \in H$, $\iota(a) \in H$.
    \end{enumerate}
\end{definition}

\begin{definition}
    \label{definition : Maximal}
    \lean{AlgebraInLean.Maximal}
    \leanok
    \uses{definition : Subgroup}
    For any group $G$, the \textit{maximal} subgroup $H$ is the group that contains every element of $G$.
\end{definition}

\begin{definition}
    \label{definition : Minimal}
    \lean{AlgebraInLean.Minimal}
    \leanok
    \uses{definition : Subgroup}
    For any group $G$, the \textit{minimal} subgroup $H$ is the group that contains exclusively the identity of $G$.
\end{definition}
