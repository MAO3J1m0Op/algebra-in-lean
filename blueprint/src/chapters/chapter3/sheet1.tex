\section{Defining the Subgroup}

\begin{definition}
    \label{definition : Subgroup}
    \lean{Defs.Subgroups.Subgroup}
    \leanok
    \uses{definition : group}
    Let $G$ be a group. A \textit{subgroup} $H$ of $G$ is a subset that is itself a group under the group operation of $G$. This requires that the subgroup satisfies three operations.

    \begin{enumerate}
        \item The identity of $G$ is contained within $H$.
        \item $H$ must be closed under $G$'s group operation: for any $a,b \in H$, $\mu(a,b) \in H$.
        \item $H$ must be closed under $G$'s inversion operation: for any $a \in H$, $\iota(a) \in H$.
    \end{enumerate}
\end{definition}

\begin{definition}
    \label{definition : Subgroup_Maximal}
    \lean{Defs.Subgroups.Maximal}
    \leanok
    \uses{definition : Subgroup}
    For any group $G$, the \textit{maximal} subgroup $H$ is the group that contains every element of $G$.
\end{definition}

\begin{definition}
    \label{definition : Subgroup_Minimal}
    \lean{Defs.Subgroups.Minimal}
    \leanok
    \uses{definition : Subgroup}
    For any group $G$, the \textit{minimal} subgroup $H$ is the group that contains exclusively the identity of $G$.
\end{definition}

\begin{theorem}
    \label{theorem : Subgroup_ext}
    \lean{Defs.Subgroups.ext}
    \leanok
    \uses{definition : Subgroup}
    Two subgroups $H$ and $K$ are equal in the same way that two sets are equal: if $H \subset K$ and $K \subset H$, then $H = K$.
\end{theorem}

\begin{proposition}
    \label{proposition : Subgroup_Criterion}
    \lean{Defs.Subgroups.Subgroup_Criterion}
    \leanok
    \uses{definition : Subgroup}
    Let $G$ be a group. A \textit{subset} $H$ of $G$ is a \textit {subgroup} $\iff$
    \begin{enumerate}
        \item $H \neq \emptyset$
        \item $\forall x, y \in H, \mu x ( \iota y) \in H$
    \end{enumerate}
\end{proposition}

\begin{definition}
    \label{definition : subgroup_trans}
    \lean{Defs.Subgroups.subgroup_trans}
    \leanok
    \uses{definition : Subgroup}
    Let $G$, $H$, and $K$ be groups such that $K \leq H$ and $H \leq G$. Then $K \leq G$.
\end{definition}

\begin{theorem}
    \label{theorem : sgp_trans}
    \lean{Defs.Subgroups.sgp_trans}
    \leanok
    \uses{definition : Subgroup}
    Let $G$ be a group. For $J$, $K$, $L \leq G$ then if $J \leq K$ and $K \leq L$ then $J \leq L$.
\end{theorem}

\begin{definition}
    \label{definition : subgroup_intersection}
    \lean{Defs.Subgroups.subgroup_intersection}
    \leanok
    \uses{definition : Subgroup}
    Let $H$ and $K$ be subgroups of group $G$. Then $H \cap K$ is a subgroup of $G$.
\end{definition}
