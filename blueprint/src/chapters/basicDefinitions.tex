\chapter{Basic Definitions}

\begin{definition}[Magma]
    \label{definition : magma}
    \lean{Defs.Magma}
    \leanok
    A magma consists of a set $G$ equipped with a single binary operation $\mu$.
    No other properties are imposed.
\end{definition}

\begin{definition}[Semigroup]
  \label{definition : semigroup}
  \lean{Defs.Semigroup}
  \leanok
  \uses{definition : magma}
  A semigroup $G$ is a magma where the operation $\mu$ is associative:
  For all $a, b, c \in G$, we have $\mu(a , \mu (b , c)) = \mu(\mu(a, b), c)$
\end{definition}

\begin{definition}[Monoid]
  \label{definition : monoid}
  \lean{Defs.Monoid}
  \leanok
  \uses{definition : semigroup}
  A monoid $G$ is a semigroup that contains an identity element $e$ that satisfies the condition:
  for all $a \in G\, , \mu(a, e) = a = \mu(e, a)$.
\end{definition}

\begin{definition}[Commutative Monoid]
  \label{definition : commutative monoid}
  \lean{Defs.CommMonoid}
  \leanok
  \uses{definition : monoid}
  A commutative monoid $G$ is a monoid where the binary operation $\mu$ is commutative:
  for all $a, b \in G\, , \mu (a, b) = \mu(b, a)$
\end{definition}

\begin{definition}[Group]
  \label{definition : group}
  \lean{Defs.Group}
  \leanok
  \uses{definition : monoid}
  A group $G$ is a monoid along with an inverse map $\iota : G \to G$ such that
  for all $a \in G\, , \mu ((\iota~a), a) = e$
\end{definition}

\begin{definition}[Abelian Group]
  \label{definition : abelian group}
  \lean{Defs.AbelianGroup}
  \leanok
  \uses{definition: group, definition: commutative monoid}
  An abelian group $G$ is a group where the binary operation is commutative:
  for all $a, b \in G\, , \mu (a, b) = \mu(b, a)$
\end{definition}
